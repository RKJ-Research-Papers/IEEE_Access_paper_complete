\section{Conclusion}

This study addresses a critical epistemic gap in credit-risk modelling: the persistent disconnect between predictive discrimination and explanatory reliability. While modern ensemble methods such as Bagged Neural Networks (BagNN) and Boosting establish strong predictive baselines in standard benchmarks, our results show that predictive success alone provides no assurance that a model’s explanations are trustworthy or decision-relevant.

Applying the proposed unified predictive--explanatory framework reveals a structural paradox at the core of contemporary explainable AI practice. Despite achieving robust AUC scores ($>0.80$), many models produce feature attributions with Sanity Ratios close to 1.015, indicating explanatory signals barely distinguishable from random noise. This demonstrates that reliance on predictive metrics alone masks the fragility of post-hoc explanations and risks overconfidence in models whose internal reasoning is weakly supported by data. In practice, explanation quality varies independently of predictive accuracy.

By explicitly diagnosing attribution instability through a dual-selector mechanism and reliability scoring, the framework shifts explainability from descriptive storytelling toward empirically grounded validation. Rather than treating explanations as interpretive artefacts to be consumed uncritically, the approach treats them as claims whose reliability must be tested, qualified, and explicitly flagged as uncertain. This reframing is essential for regulated credit-risk environments, where transparency, challengeability, and auditability are as important as predictive performance.

More broadly, the framework demonstrates how predictive modelling, attribution robustness, and constrained generative explanation can be integrated into a single governance-oriented workflow. By embedding reliability diagnostics directly into human-readable explanations, the approach supports informed decision-making without overstating model certainty and provides financial institutions with a transparent pathway to align advanced machine-learning systems with Basel model-risk management expectations, while establishing a foundation for future research that treats explainability as a scientifically testable component of model validity rather than a cosmetic add-on.
